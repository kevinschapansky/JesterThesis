\chapter{Results}

The primary validation of Jester is the existence of a simple process for creating thin sensor wrappers and simple data fusion modifications. Chapter \ref{chap:sample_app} documented the process of wrapping different skeletal tracking APIs and modifying the data fusion module. A huge bulk of the already relatively small Jester specific work went into creating the sensor wrappers. Ideally, these wrappers could be created by third party developers and then shared, so new application developers would just have to get the already existing wrapper modules from other developers; eliminating a big portion of the setup time.

Human interaction applications have an inherent requirement of being soft real time. A system slow down will not cause catastrophic failure, but users are hesitant to put up with high latency interfaces since modern computers are currently very responsive. So, Jester's internal data flow must not add any significant overhead. Deciding what exactly "real time" means is not necessarily straight forward, but for the purposes of this evaluation a metric must be chosen. Video games typically require at least 30 fps in order to appear to be fluid. This thesis defines real time as meaning that the Jester data intake and fusion process can loop at a minimum of 30 hertz. This chapter outlines the different parts of the Jester data acquisition and fusion process to validate that there is no significant performance bottleneck. All tests are run on a system with an Intel Core 2 Duo 2.6 GHz processor and 4GB of system memory.

\section{Jester Sensor Input and Core Performance}

The Jester core is discussed in Section \ref{sec:jester_core}. Its data flow is essentially fixed and will not change based on the requirements of the application. Since every new sensor reading that is generated must be fed through the core to reach the skeletal model speed is essential; especially since the function calls to get data to the data fusion module are blocking. The wall clock time for a new set of joint data to be sent from the sensor wrapper, to the controller, to the data fuser, and then transformed into bones is used. Joint datasets are more computationally expensive than bone sets because they must be translated into bones before they can be used. The average times to feed joint data through the Jester core is shown in Table \ref{tab:core}.

\begin{table}[h]
\begin{tabular}{llll}
Data Step                        & Average Time (ms) & Max Time (ms) & Min Time (ms) \\
Leap Data -\textgreater Fuser    & 0.134             & 6.108         & 0.026             \\
Carmine Data -\textgreater Fuser & 0.125             & 11.58         & 0.06              \\
Skeleton Update                  & 0.049             & 3.74          & 0.004
\end{tabular}
\caption{Jester core data flow performance.}
\label{tab:core}
\end{table}

\section{Data Fuser Performance}

The data fusion module is the most computationally expensive part of the Jester framework. It handles all of the functionality described in Section \ref{sec:fuser_des_impl}. Different fusion algorithms and filtering methods could have a profound impact on the fusion performance. A fuser that simply passes new data directly to the skeleton would finish almost instantaneously while a fuser that does extensive inverse kinematics or filtering could be so slow that it is no longer real time. The performance of the data fusion unit with the small modification described in Section \ref{sec:fuser_impl} with and without the filtering module from Section \ref{sec:filter_impl} are shown in Table \ref{tab:fuser}.

\begin{table}[h]
\begin{tabular}{llll}
Data Step         & Average Time (ms) & Max Time (ms) & Min Time (ms) \\
Data Fusion       & 0.133             & 6.819         & 0.025         \\
Swap Check        & 0.013             & 3.427         & 0.001         \\
Double Exp Filter & 0.02              & 4.698         & 0.001        
\end{tabular}
\caption{Data fusion implementation performance.}
\label{tab:fuser}
\end{table}

\section{Combined Performance}

In order to validate the refresh rate of the Jester framework and sample implementation, all of the parts must be evaluated together. The cumulative system time and the resulting refresh rate is shown in Table \ref{tab:total}. Even if all of the worst case times are added, which is highly unlikely in reality, the system still functions at 27.5 hertz. The maximum times in all of the steps are rather extreme outliers. This is most likely because the performance statistics are measured as wall clock time. Thus, there is a large reported performance discrepancy when the operating system de-schedules Jester to perform background tasks. The average performance, gathered over 5000 system iterations, provides a more realistic view of Jester's efficiency. On average, the entire Jester data intake and fusion processes can complete at over 2000 hertz. Therefore, Jester does not add any significant system overhead and will not prevent a client application from functioning in ``real time".

\begin{table}[h]
\begin{tabular}{llll}
             & Average Case & Worst Case & Best Case \\
Total Time   & 0.474        & 36.37      & 0.117     \\
Refresh Rate & 2110 Hz      & 27.5 Hz    & 8547 Hz  
\end{tabular}
\caption{Cumulative system performance times and the resulting refresh rate.}
\label{tab:total}
\end{table}
